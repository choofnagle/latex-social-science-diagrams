\documentclass{article}
\usepackage{tikz,graphicx}
\usetikzlibrary{positioning,matrix,calc}
\begin{document}
\thispagestyle{empty}
% \begin{center}

\begin{figure}
\tikz{
% Define style
\tikzstyle{block}=[draw,outer sep=0pt, inner sep=0pt, minimum width=4cm, minimum height=2cm, text width=3.8cm, fill=gray!15, align=left]

\matrix (t) at (0,0) [
    matrix of nodes,
    nodes in empty cells, 
    nodes={minimum size=1cm, anchor=north,align=left,text width=5.5cm,inner ysep=2mm},
    row sep=0cm,
    column sep=0.3cm
  ] {Significant kinetic or cyber aggression destroying lives or property & International Humanitarian Law’s notions of military objectives, civilian distinction, proportionality \\
  Election interference, disinformation, CNE of government systems, government-sponsored industrial espionage & Domestic criminal law, sanctions, indictments of identified hackers (Kello claims these under-deter) \\
Ordinary social/economic tension from markets, espionage, ordinary crime  & Domestic criminal law (can be effective) \\
  };

\draw[thick, latex-latex,teal!60!blue] ($(t-3-1.south west)+(-.2,-.2)$)--coordinate(o) node[pos=0,below=1mm,black]{Total Peace} ($(t-1-1.north west)+(-.2,.2)$);
\draw[thick,teal!50!blue] ($(t-2-1.north west)+(-.2,0)$)--($(t-2-2.north east)+(.2,0)$);
\draw[thick,dashed,teal!60!blue] ($(t-3-1.north west)+(-.2,0)$)--($(t-3-2.north east)+(.2,0)$);

\node[above=3mm] (lc) at(t-1-2.north) {Legal controls};
\node[] at(lc-|o) {Total War};
\node[left=6mm] (lw) at(t-1-1.west) {Lawful War};
\node[anchor=west,align=left] at(t-2-1.west-|lw.west) {Kello’s\\Unpeace};
\node[anchor=west,align=left] at(t-3-1.west-|lw.west) {Normal State\\Competition};

}    
% \end{center}

\caption{Cybersecurity theorist Lucas Kello embraces defend forward as a good first step, but argues that Western nations go farther: they should ``punish backward'' through a system of ``punctuated deterrence.'' Kello observes that ``unpeace'' or below-threshold conflict allows nations to levy low-level cyberattacks that have a cumulative strategic effect. That is, countries can achieve strategic effects through disparate attacks, none of which constitutes a use of force. To counter these attacks, Kello argues that ``nations that  suffer  sustained  acts  of  unpeace  should  punish  them  as  campaigns  that inflict \textit{cumulative damage} rather than as individual actions.''  In other words, nations must keep a balance sheet of cyber and other offenses against their sovereignty, and commit to punish broadly when the sheet reaches some cumulative threshold of aggression. The US indeed has shifted from a purely deterrence-by-denial strategy to one with more punishment, in the forms of indicting foreign hackers and in persistent engagement. Kello's work suggests that nations should go further and commit to punishments such as broad sanctions regimes, even if these are economically costly to impose. See \textsc{Striking Back: The End of Peace in Cyberspace - And How to Restore It}, Yale Univ. Press 2022}
\end{figure}


\end{document}
